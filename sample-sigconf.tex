\documentclass[sigconf]{acmart}

\usepackage{booktabs} % For formal tables


% Copyright
%\setcopyright{none}
%\setcopyright{acmcopyright}
%\setcopyright{acmlicensed}
\setcopyright{rightsretained}
%\setcopyright{usgov}
%\setcopyright{usgovmixed}
%\setcopyright{cagov}
%\setcopyright{cagovmixed}


% DOI
\acmDOI{10.1145/3139937.3139939}

% ISBN
\acmISBN{978-1-4503-5396-0/17/11}

%Conference
\acmConference{IoT-S\&P'17}{November 3, 2017}{Dallas, TX, USA} 
\acmYear{2017}
\copyrightyear{2017}


%%\acmArticle{4}
\acmPrice{}


\begin{document}
\title{Cleartext Data Transmissions in Consumer IoT Medical Devices}
%%\subtitle{Extended Abstract}
%%\subtitlenote{The full version of the author's guide is available as
%%  \texttt{acmart.pdf} document}


\author{Daniel Wood}
\affiliation{%
  \institution{Princeton University}
  \city{Princeton} 
  \state{New Jersey} 
}
\email{dewood@princeton.edu}

\author{Noah Apthorpe}
\affiliation{%
  \institution{Princeton University}
  \city{Princeton} 
  \state{New Jersey} 
}
\email{apthorpe@cs.princeton.edu}

\author{Nick Feamster}
\affiliation{%
  \institution{Princeton University}
  \city{Princeton} 
  \state{New Jersey} 
}
\email{feamster@cs.princeton.edu}


% The default list of authors is too long for headers}
\renewcommand{\shortauthors}{D. Wood et al.}

\begin{abstract}
This paper introduces a method to
capture and analyze transmitted data from medical IoT devices to detect plaintext
payload and metadata that reveals a user's
medical conditions and
behavior. 
The research follows a three-step approach involving collection, plaintext detection
and analysis, and metadata analysis. 
We analyze four popular consumer medical IoT devices and present 
results from our analysis, including one smart medical device that leaks sensitive
health information in plaintext. 
Towards providing users more visibility into the data that these devices are transmitting
and to enable automatic detection of sensitive data transmission,
we have developed a traffic capture and analysis system that seamlessly integrates
with a user's home network and offers
a user-friendly interface for consumers to monitor and visualize data transmissions
of IoT devices in their home; we have released this software open-source and also
recently presented this to several policymakers who are studying vulnerabilities
in IoT devices, including members of the United States Congress.
\end{abstract}

%
% The code below should be generated by the tool at
% http://dl.acm.org/ccs.cfm
% Please copy and paste the code instead of the example below. 
%
\begin{CCSXML}
<ccs2012>
<concept>
<concept_id>10002978</concept_id>
<concept_desc>Security and privacy</concept_desc>
<concept_significance>500</concept_significance>
</concept>
<concept>
<concept_id>10003120.10003138</concept_id>
<concept_desc>Human-centered computing~Ubiquitous and mobile computing</concept_desc>
<concept_significance>300</concept_significance>
</concept>
</ccs2012>
\end{CCSXML}

\ccsdesc[500]{Security and privacy}
\ccsdesc[300]{Human-centered computing~Ubiquitous and mobile computing}


\keywords{Personal health information; medical devices; Internet of Things; privacy}


\maketitle

\input{body}

\bibliographystyle{ACM-Reference-Format}
\bibliography{bibliography} 

\end{document}
